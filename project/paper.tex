\documentclass[12pt]{article}

% This first part of the file is called the PREAMBLE. It includes
% customizations and command definitions. The preamble is everything
% between \documentclass and \begin{document}.

\usepackage[margin=1in]{geometry}  % set the margins to 1in on all sides
\usepackage{graphicx}              % to include figures
\usepackage{amsmath,bm}            % great math stuff
\usepackage{amsfonts}              % for blackboard bold, etc
\usepackage{amsthm}                % better theorem environments
\usepackage{breqn}
\usepackage{listings}
\usepackage{hyperref}
\usepackage{tikz}
\usepackage{tikzscale}
\usepackage{pgfplots}
\usepackage[linesnumbered,ruled,vlined,algosection]{algorithm2e}
\usepackage{parskip} 			   % no paragraph indentation
\usetikzlibrary{arrows,automata,decorations.pathreplacing,math,arrows,automata}
\usepackage{subcaption}

\newlength\marincrease
\makeatletter
\newenvironment{Walgo}[2][htbp]
  {\renewcommand{\@algocf@start}{%
    \setlength\marincrease{#2}
  \@algoskip%
  \begin{lrbox}{\algocf@algobox}%
  \begin{minipage}{\textwidth}
  \setlength{\algowidth}{\hsize}%
  \vbox\bgroup% save all the algo in a box
  \hbox to\algowidth\bgroup\hbox to \algomargin{\hfill}\vtop\bgroup%
  \ifthenelse{\boolean{algocf@slide}}{\parskip 0.5ex\color{black}}{}%
  % initialization
  \addtolength{\hsize}{-1.5\algomargin}%
  \let\@mathsemicolon=\;\def\;{\ifmmode\@mathsemicolon\else\@endalgoln\fi}%
  \raggedright\AlFnt{}%
  \ifthenelse{\boolean{algocf@slide}}{\IncMargin{\skipalgocfslide}}{}%
  \@algoinsideskip%
%   \let\@emathdisplay=\]\def\]{\algocf@endline\@emathdisplay\nl}%
  }%
\renewcommand{\@algocf@finish}{%
  \@algoinsideskip%
  \egroup%end of vtop which contain all the text
  \hfill\egroup%end of hbox wich contains [margin][vtop]
  \ifthenelse{\boolean{algocf@slide}}{\DecMargin{\skipalgocfslide}}{}%
  %
  \egroup%end of main vbox
  \end{minipage}
  \end{lrbox}%
  \makebox[\linewidth][c]{\algocf@makethealgo}% print the algo
  \@algoskip%
  % restore dimension and macros
  \setlength{\hsize}{\algowidth}%
  \lineskip\normallineskip\setlength{\skiptotal}{\@defaultskiptotal}%
  \let\;=\@mathsemicolon%  
  \let\]=\@emathdisplay%
}%
  \begin{algorithm}[#1]}
  {\end{algorithm}}
\makeatother

% various tikz defs
\makeatletter
\newcommand{\gettikzxy}[3]{%
  \tikz@scan@one@point\pgfutil@firstofone#1\relax
  \edef#2{\the\pgf@x}%
  \edef#3{\the\pgf@y}%
}
\makeatother

% various theorems, numbered by section

\newtheorem{thm}{Theorem}[section]
\newtheorem{lem}[thm]{Lemma}
\newtheorem{prop}[thm]{Proposition}
\newtheorem{cor}[thm]{Corollary}
\newtheorem{conj}[thm]{Conjecture}
\newtheorem{defi}[thm]{Definition}

\DeclareMathOperator{\id}{id}

\newcommand{\bd}[1]{\mathbf{#1}}  % for bolding symbols
\newcommand{\RR}{\mathbb{R}}      % for Real numbers
\newcommand{\ZZ}{\mathbb{Z}}      % for Integers
\newcommand{\col}[1]{\left[\begin{matrix} #1 \end{matrix} \right]}
\newcommand{\comb}[2]{\binom{#1^2 + #2^2}{#1+#2}}

\lstset{ % General setup for the package
    language={[LaTeX]TeX},
    basicstyle=\footnotesize\sffamily,
    tabsize=4,
    columns=fixed,
    keepspaces,
    commentstyle=\color{red},
    keywordstyle=\color{blue},
    xleftmargin=.1\textwidth,
    xrightmargin=.1\textwidth
}

% argument #1: any options
\newenvironment{customlegend}[1][]{%
    \begingroup
    % inits/clears the lists (which might be populated from previous
    % axes):
    \csname pgfplots@init@cleared@structures\endcsname
    \pgfplotsset{#1}%
}{%
    % draws the legend:
    \csname pgfplots@createlegend\endcsname
    \endgroup
}%
\def\addlegendimage{\csname pgfplots@addlegendimage\endcsname}
\pgfkeys{/pgfplots/number in legend/.style={%
        /pgfplots/legend image code/.code={%
            \node at (0.125,-0.0225){#1}; % <= changed x value
        },%
    },
}
\pgfplotsset{
every legend to name picture/.style={west}
}


\begin{document}

\nocite{*}


\title{Topics in Algorithms}

\author{Viktor Hansen \& Mehdi Nadif}

\maketitle

\begin{abstract}
  This comprises our project covering the first half of the paper "Approximation Distance Oracles with Improved Query Time" by Christian Wulff-Nilsen \cite{Wulff2013}. This project attempts to elaborate and provide intuition on the $\lg k$-query time improvement by providing explanations and visualisations of proofs.
\end{abstract}

\pagebreak

\section{Overview}
Given an undirected graph $G$ with non-negative edge weights, $m$ edges, $n$ vertices and some $k \in \mathbb{Z} \geq 2$, Thorup and Zwick showed that a $(2k-1)$-approximate distance oracle for $G$ of size $O(kn^{1+1/k})$ and $O(k)$ query time can be constructed in $O(kmn^{1/k})$ expected time \cite{Thorup2005}. \cite{Wulff2013} improves this query time to $\lg k$, while preserving stretch, space and preprocessing time. % Specifically $O( \min \left\{ kmn^{1/k}, \sqrt{k}m+kn^{1+c/\sqrt{k}} \right\})$ time for some constant $c$.

The query time reduction is made possible by observing that binary search can be performed on the iteration sequence of the $dist_k(u)$ algorithm described in \cite{Thorup2005}. This is done by precomputing a secondary data structures $\mathcal{T}_u$ for each $u \in V$ in $O(k)$ time. Each $\mathcal{T}_u$ contains information about which subsequences of $0, \hdots, k-1$ that make the modified $dist_k(u,i)$ algorithm (see Alg. \ref{alg:distk}) terminate while maintaining a  stretch of $(2k-1)$. Such subsequences are called $(u,v)$-feasible. Specifically, the subsequences in $T_u$ are \textit{canonical}. $(u,v)$-feasibility and canonicity is defined and discussed in 3.

% Assumptions about the reader

\section{Definitions}
The following definitions taken directly from \cite{Thorup2005} and \cite{Wulff2013} are used throughout this project and are listed here for brevity and coherency:
\begin{itemize}
\item $d_{G}(u,v)$: the shortest path distance between $u$ and $v$.
\item $A_i = \left\{ a \in A_{i-1} \; | \; a \text{ sampled with prob. } n^{-1/k} \text{ uniformly, i.i.d.} \right\}$
\item $(V = A_0) \supseteq A_1 \supseteq \hdots \supseteq (A_k = \emptyset)$
\item $p_i(u) = \min_{v \in A_i} \left\{ d_{G}(u,v) \right\}$
\item $B_u=\bigcup_{i=0}^{k-1}\left\{ v \in A_i - A_{i+1} \; | \; d_{G}(u,v) < d_{G}(u, p_{i+1}(u)) \right\}$, the bunch of $u$.
\item $d_G(u,p_k(u))=\infty$
\end{itemize}
Thorup and Zwick showed the property about bunches seen in Lemma \ref{lem:onedeltaperiteration}. The lemma is visualized in Fig. \ref{fig:lemma1vis}, and follows from the definitions of bunches. That $A_{k-1} \subset B_u$ and $p_{k-1}(v) \in B_u$ also readily follows from the definition of $A_{k}$ and $d_G(u,p_k(u))$ shown above:
\\
\begin{lem}
Let $u,v \in V$ be distinct vertices $0 \leq i < k-1$. If $p_i(v) \not\in B_{u}$ then $d_G(u, p_{i+1}(u)) \leq d_G(u, p_i(v))$. Furthermore $A_{k-1} \subset B_u$ and $p_{k-1}(v) \in B_u$. \label{lem:onedeltaperiteration}
\end{lem}
This lemma is readily used for bounding $d_G(u,v)$ by successively applying the triangle inequality at each iteration of $dist_k(u,v)$ to arrive at the $(2k-1)$-stretch bound in \cite{Thorup2005}.

\begin{figure}[!hbt]
\centering
    \includegraphics[width=0.5\textwidth]{include/tikz/lemma1.tikz}%
\caption{Visualization of Lemma \ref{lem:onedeltaperiteration}. The dotted green path corresponds to the circle defined by $u$ and the radii $d_G(p_i(u))$, which $p_{i+1}(v)$ is clearly outside. \label{fig:lemma1vis}}
\end{figure}

\section{Oracle with $O(\lg k)$ query time}
As aforementioned, binary search on the iteration sequence of the Thorup-Zwick oracle is performed to drive down the query time to $\lg k$. Starting with the iteration sequence $\mathcal{I}=1,\hdots,k-1$, the idea is to identify $r=O(\lg k)$ subsequences $(\mathcal{I}_1=\mathcal{I}) \supset \mathcal{I}_1 \supset \hdots \supset \mathcal{I}_r$, where for $2 \leq j \leq t$ we have $|\mathcal{I}_{j-1}| \geq 2|\mathcal{I}_j|$. We wish to maintain the invariant that $dist_k(u,v,i)$ applied to the first element of any subsequence $\mathcal{I}_j$ outputs the stretch $(2k-1)$ estimate in $|\mathcal{\mathcal{I}_j}|$ time.
\\
\begin{defi}
For vertices $u,v \in V$ an index $j \in \mathcal{I}$ is $(u,v)$-terminal if
\begin{enumerate}
\item $j = k-1$ or
\item $j < k-1$ is even and $p_j(u) \in B_v$ or $p_{j+1}(v) \in B_u$ 
\end{enumerate}
We note that $j$ attains a $(u,v)$ terminal index, the algorithm terminates in the current or next iteration. \label{def:uvterm}
\end{defi}
The definition of $(u,v)$-terminality captures the essence of when $dist_k(u,v)$ terminates, i.e. when $p_j(u) \in B_v$. This happens either when 1) $j=k-1$, in which case, trivially, $p_{k-1}(u) \in B_v$ or 2) when $j < k-1$ is even and $p_j(u) \in B_v$ or $p_{j+1}(v) \in B_u$ (in this case $dist_k(u,v)$ terminates either in iteration $j$ or $j+1$), see Alg. \ref{alg:distk}.
\\
\begin{defi}
A subsequence $\mathcal{I'}=i_1, \hdots, i_2$ of $\mathcal{I}$ is $(u,v)$-feasible if
\begin{enumerate}
\item $i_1$ is even and
\item $d_{G}(p_{i_1}(u),u) \leq i_1 \cdot d_G(u,v)$ and
\item $i_2$ is $(u,v)$-terminal
\end{enumerate}
\label{def:uvfeas}
\end{defi}
Furthermore, the definition of $(u,v)$-feasibility 

\begin{Walgo}[ht]{20pt}
  \DontPrintSemicolon
  \SetKwRepeat{Do}{do}{while}
  \SetKwInOut{Input}{Input}
  \SetKwInOut{Output}{Output}
  \Indm
    \Indp
    \BlankLine
    $w \leftarrow p_i(u); j \leftarrow i$
    
    \While{$w \not\in B_v$}{
      $j \leftarrow j + 1$
      
      $(u,v) \leftarrow (v,u)$
      
      $w \leftarrow p_j(u)$
    }
    \Return{$d_G(w,u) + d_G(w,v)$}
    
    \caption{$dist_k(u,v,i)$}
    \label{alg:distk}
\end{Walgo}

\begin{Walgo}[ht]{20pt}
  \DontPrintSemicolon
  \SetKwRepeat{Do}{do}{while}
  \SetKwInOut{Input}{Input}
  \SetKwInOut{Output}{Output}
  \Indm
    \Indp
    \BlankLine
    \If{$i_2-i_1 \leq \lg k$}{
    	\Return{$dist_k(u,v,i_1)$}
	}
    
    \textbf{let} $i$ be the middle even index $i_1,\hdots,i_2$
    
    \textbf{let} $i$ be the (precomputed) even index in $i_1,\hdots,i-2$ maximizing $\delta_j(u)$
    
    \If{$p_j(u) \not\in B_v$ \textbf{and} $p_{j+1} \not\in B_u$}{
    	\Return{$bdist_k(u,v,i,i_2)$}
    }
    \Else{
    	\Return{$bdist_k(u,v,i_1,j)$}
    }
    \caption{$bdist_k(u,v,i_1,i_2)$}
    \label{alg:bdistk}
\end{Walgo}

\begin{lem}
Let $i_1, \hdots, i_2$ be a $(u,v)$-feasible subsequence. Then $dist_k(u,v,i_1)$ gives a $2k-1$ approximation of $d_G(u,v)$ in time $O(i_2-i_1)$.
\paragraph{Proof} By assumption $i_2$ is $(u,v)$-terminal. Therefore, $p_{i_2}(u) \in B_v$ or $p_{i_2 + 1}(v) \in B_u$. Thus $dist_k(u,v,i_1)$ terminates when $j=O(i_2)$ in time $O(i_2-i_1)$.

Furthermore $d_{G}(p_{i_1}(u),u) \leq i_1 \cdot d_G(u,v)$. When $p_j(u) = w \not\in B_v$
we have $d_G(v,p_{j+1}(v)) \leq d_G(v,p_{j}(u)) \leq d_G(v,u) + d_G(u,p_j(u))$ by Lemma \ref{lem:onedeltaperiteration} and the triangle inequality. Hence each iteration of $dist_k(u,v,i)$ increases $d_G(u,w)$ by at most $d_G(u,v)$. At termination we have
\begin{align*}
d_G(u,w) + d_G(w,v) &\leq 2d_G(u,w) + d_G(u,v) \\
&\leq 2(i_1 \cdot d_G(u,v) + (i_2 - i_1)d_G(u,v)) + d_G(u,v) \\
&\leq 2 \cdot i_2 \cdot d_G(u,v) + d_G(u,v) \\
&\leq (2 \cdot i_2 + 1)d_G(u,v) \\
&\leq (2 (k-1) + 1)d_G(u,v) \\
&\leq (2k-1) d_G(u,v)
\end{align*}
\end{lem}
% Inkluder illustration - forklarende tekst "By lemma 2.1 and the triangle inequality"

% Lemma 3.2 her - Hele sekvensen, svarende til at køre Thorup-Zwick i=1...k

% Lemma 3.3 her - "skip hver anden" - gør analysen symmetrisk - lav illustration (projicer punkter ned på reelle akse og indtegn afstande). Forklar bevis.

\begin{lem}
Let $i_1, \hdots, i_2$ be a $(u,v)$-feasible subsequence and let $i$ be even, $i_1+2 \leq i \leq i_2-2$. Let $j$ be an even index in subsequence $i_1, \hdots ,i-2$ that maximizes $\delta_j(u)$. If $p_j(u) \not\in B_v$ and $p_j+1(v) \not\in B_u$ then $i, \hdots , i_2$ is $(u,v)$-feasible. Otherwise, $i_1, \hdots , j$ is $(u,v)$-feasible.

\paragraph{Proof} If $p_j(u) \in B_v$ or $p_j+1(v) \in B_u$, then $j$ is $(u,v)$-feasible, and so is $i_1, \hdots, j$. Assume that $p_j(u) \not\in B_v$ and $p_j+1(v) \not\in B_u$. Then, by Lemma \ref{lem:onedeltaperiteration}, $d_G(v,p_{j+1}(v)) \leq d_G(v,p_{j}(u))$ and $d_G(u,p_{j+2}(u)) \leq d_G(u,p_{j+1(v)})$. Applying the triangle inequality twice, cfr. Fig. \ref{fig:twoiterations}, we get 
\begin{align*}
d_G(u,p_{j+2}(u) &\leq d_G(u,p_{j+1}(v)) &(\text{Lemma \ref{lem:onedeltaperiteration}}) \\
&\leq d_G(u,v) + d_G(v, p_{j+1}(v)) &(\text{Triangle Inequality})\\
&\leq d_G(u,v) + d_G(v, p_{j}(u)) &(\text{Lemma \ref{lem:onedeltaperiteration}})\\
&\leq 2d_G(u,v) + d_G(u, p_{j}(u))&(\text{Triangle Inequality})
\end{align*}
so $\delta_j(u) = d_G(u,p_{j+2}(u)) - d_G(u,p_j(u)) \leq 2d_G(u,v)$. Let $\mathcal{I}'$ be the set of even indices $i_1, i_1+2, i_1+4, \hdots, i-2$. Since $i_1, \hdots, i_2$ is $(u,v)$-feasible, $d_G(u,p_{i_1}(u)) \leq i_1 \cdot d_G(u,v)$. By the choice of $j$,
\begin{align*}
d_G(u,p_{i}(u)) &= d_G(u,p_{i_1}(u)) + \sum_{j' \in \mathcal{I'}}\delta_{j'}(u) \\
&\leq d_G(u,p_{i_1}(u)) + \left| \mathcal{I'} \right| \max_{j' \in \mathcal{I'}}\delta_{j'}(u) \\
&= d_G(u,p_{i_1}(u)) + \frac{i-i_1}{2} \delta_{j}(u) \\
&\leq i_1 \cdot d_G(u,v) + (i-i_1)d_G(u,v) \\
&= i \cdot d_G(u,v)
\end{align*}
Hence, since $i_1, \hdots, i_2$ is $(u,v)$-feasible, so is $i, \hdots, i_2$.  \label{lem:uvfeasibility}
\end{lem}

\begin{figure}[!hbt]
  \centering
  \subcaptionbox{$(j)$th iteration\label{fig:j1}}{
    \includegraphics[width=.45\textwidth]{include/tikz/jth_iteration.tikz}%
  }
  \hfill
  \subcaptionbox{$(j+1)$th iteration\label{fig:j2}}{
    \includegraphics[width=.45\textwidth]{include/tikz/j1th_iteration.tikz}%
  }
  \caption{Two consecutive iterations of the while loop in $dist_k$. Both sampled points of $u$, $p_j(u)$ in Fig. \ref{fig:j1} and $p_{j+1}(u)$ in Fig. \ref{fig:j2}, are not in $B_v$ and therefore $j$ is not $(u,v)$-terminal. An upper bound of $2d_{G}(u,v)+d_G(u,p_j(u))$ is derived on $d_G(u,p_{j+2}(u))$ in Lemma \ref{lem:uvfeasibility} by applying Lemma \ref{lem:onedeltaperiteration} and the triangle inequality twice. \label{fig:twoiterations}}
\end{figure}

\begin{figure}
\centering
    \includegraphics[width=1\textwidth]{include/tikz/distance_real_line.tikz}%
\caption{Visualization of the distance $d_G(u,p_j(u))$ and $\delta_j(u)$. The green lines correspond to the circles defined by $u$ and the radii $d_G(p_l(u))$ for $l \in \left\{ i_1, j, j+2, i-2, i, i_2 \right\}$.}
\end{figure}

\begin{figure}[h]
\begin{tikzpicture}[level distance=1.5cm,
  level 1/.style={sibling distance=3cm},
  level 2/.style={sibling distance=1.5cm}]
  \node[color=red] {$I'$}
    child[color=red] {node {$[0, ..., i]$}
    child[color=black,thick] {node[color=cyan] {$I^j$}}
      child[dashed] {node {$p(l_1)$}
      	child[solid, left=0.1cm, color=black] {node[color=cyan] {$l_1=[i_1, i_1+2]$}}
      	child[solid,color=black] {node[color=cyan] {$[i_1+2,i_1+4]$}}
      }
    }
    child[right=0.1cm, color=red] {node {$[i,...,k-1]$}
    child[color=black] {node[color=cyan] {$I^k$} 
    						child[dotted, color=black] {node {...}}
    						child[dotted, color=black,left=0.1cm] {node {...}}
    					}
      child[dashed] {{node {$p(l_2)$}
       	child[solid,color=black] {node[color=cyan] {$[i_2-4,i_2-2]$}}
      	child[right=1cm,solid,color=black] {node[color=cyan] {$l_2=[i_2-2, i_2]$}}
    }}};
    
\begin{customlegend}[legend cell align=left, %<= to align cells
legend entries={ % <= in the following there are the entries
$P$: path between $p(l_1)$ and $p(l_2)$,
 nodes belonging to $X$.
},
legend style={at={(0,2)},font=\footnotesize}] % <= to define position and font legend
% the following are the "images" and numbers in the legend
    \addlegendimage{solid,red}
    \addlegendimage{area legend,cyan, fill=cyan}
\end{customlegend}
\end{tikzpicture}
\caption{A tree $T_u$ representing the recursion of $bdist_k$ of $u$, each node corresponding to the interval of a recursive call. The root node $I'$ consists of all even indices between $0$ and $k-1$.\label{tree}}
\end{figure}

\begin{thm}
For an integer $k\geq 2$, a $(2k - 1)$-approximate distance oracle of G of size $O(kn^{1+1/k})$ and $O(\log k)$ query time can be constructed in $O\left( \min \left\lbrace kmn^{1+\frac{1}{k}}, \sqrt{k}m + kn^{1+c/k}\right\rbrace\right)$ time for some constant $c$.
\end{thm}
{\it Proof.} For variable $k\geq 3$, for each $u$ the bunch $B_u$ in computed in $O(kmn^{2})$ using \cite{Thorup2005}. The remaining preprocessing for $u$ is determining the necessary subsequences of $\mathcal{I}$ and the $j$ that maximized $\delta_j(u)$ for each subsequence.
The necessary subsequences are built 

To bound the time to identify these desired $j$, a simple linear search requires $O(k\log k)$ time. This can be improved to $O(k)$ by constructing of a recursive tree $\mathcal{T}_u$ for each $u$ that represents the recursion of $bdist_k$ on the subsequence $\mathcal{I}'$ consisting of all even numbers in $\mathcal{I}$. A bottom-up seach of $\mathcal{T}_u$ can identify for each canonical subsequence (canonical meaning it is a node in $\mathcal{T}$) $i_1,i_1+2,...,i_2$ an index $j$ in the subsequence that maximizes $\delta_j(u)$ in $O(k)$ time. The bottom-up traversal of $T$ requires checking at most $O(\log k)$ subsequences; consider the subsequence $i_1,..,i_2$ in seperate leaf nodes as seen in Fig. \ref{tree}. The path $P$ (red) starting in $p(l_1)$ and ending in $p(l_2)$ has $O(\log k)$ children, which we denote $X$. The union of $X$ clearly covers the entire subsequence $i_1,i_1+2,...,i_2-2,i_2$, and finding a $j$ that maximizes $\delta_j(u)$ in $X$ therefore gives a $j$ that maximizes $\delta_j(u)$ in the subsequence $i_1,...,i_2$. If $j$ is known for each node in $\mathcal{T}$, it follows that finding the desired $j$ for any canonical subsequences is bound by $O(\log k)$\\

The recursion tree for $bdist_k$ for $u$ will have $\theta(k/\log k)$ children and $\theta(k/\log k)$ non-child nodes. Computing $j$ for each subsequence in $T$ will take $O(\log k)$ time per node, so total preprocessing time for $u$ is $O(k)$.  
\bibliographystyle{plain}
\bibliography{bibliography} 

\end{document}
